\documentclass[12pt]{article}
\usepackage[utf8]{inputenc}
\usepackage{graphicx}
\usepackage{subcaption}

\usepackage[letterpaper]{geometry}
    \geometry{top=1.0in, bottom=1.0in, left=1.0in, right=1.0in}
\usepackage{setspace}
    \doublespacing


\begin{document}

\begin{center}
    {\bf \large\fontsize{15}{15} \selectfont Groovy} \\
    {Hannah Smith \& Richard Wong} \\
    {CSc372, Spring 2019}\\
\end{center}

\begin{center}
\line(1,0){460}
\end{center}

\section*{\fontsize{14}{15}\selectfont Abstract}


\section*{\fontsize{14}{15}\selectfont Introduction}


\section*{\fontsize{14}{15}\selectfont History}
Groovy was born of a desire for a dynamically typed language for the Java platform- one that combines the convenience and ease of languages like Python and Ruby with Java-like syntax and the ability to utilize the Java API and run on the Java virtual machine. In a blog post from 2003, James Strachan first revealed that this new language, referred to as "Groovy", was in development. However, it wasn't until January 2, 2007 that Groovy 1.0 was released to the public. Since 2003, Groovy has switched hands many times, with its original mastermind, Strachan, leaving before Groovy 1.0 was even officially released. As of November 2015, according to a release by the Groovy-dev mailing list, it is now a top-level project in the Apache Software Foundation. As with any programming language, Groovy has evolved over time and today even includes static compiling and type-checking, a feature added with the release of Groovy 2.0 in 2012.

\section*{\fontsize{14}{15}\selectfont Control Structures}

\section*{\fontsize{14}{15}\selectfont Data Types}

\section*{\fontsize{14}{15}\selectfont Subprograms}

\section*{\fontsize{14}{15}\selectfont Summary}



%% Works Cited:
\newpage
\begin{center}
    {\bf \large\fontsize{15}{15} \selectfont References} \\
    \line(1,0){460}
\end{center}

\vspace{0.5cm}

\noindent Subramaniam, Venkat., and Daniel H. Steinberg. \textit{Programming Groovy : Dynamic Produc- \indent tivity for the Java Developer}. Raleigh, N.C.: Pragmatic helf, 2008. Print.
\\


\end{document}

